\section{Link Analysis}

Will a wireless link from $A$ to $B$ work? What speeds can be
expected? How much transmit power is necessary and what kind of
antennae should be used? To answer these questions we do a link budget
calculation. Roughly speaking this is,
\begin{equation}
  P_r = P_t + G - L
\end{equation}
or, the received power is equal to the transmit power, plus gains, and
minus losses. We then check in the manufacturer's table to see if the
received power is sufficient to maintain a link and at what speed.

The main difficulty is in figuring out what $L$ is, because there are
lots of different sources of loss, for example:
\begin{itemize
  \item cable loss caused by electrical resistance of a feedline
    between  between the radio and the antenna
  \item insertion loss caused by the use of connectors to attach the
    feedline
  \item path loss caused by the spreading out of energy in space (this
    is your ``inverse-$r^2$'' loss)
  \item path loss related to the decreasing ability of an antenna to
    ``hear'' radio waves as the frequency increases (this loss is
    proportional to the square of the frequency)
  \item multi-path loss where two signals from
    the same transmitter follow different paths to the receiver and
    interfere destructively with one another
  \item absorption, primarily by water in the air such as rain and
    fog, but also at higher frequencies by elements such as oxygen and
    nitrogen an effect which varies depending on frequency.
\end{itemize}

To see how this works, let us take a specific example, the link from
Eigg to Kinlocheil. 
