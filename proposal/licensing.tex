\subsection{Licensing}

The reasons that we need to to move to licensed wireless links for
this project are worked out in detail in section
\ref{sec:link_analysis} and particularly the part about regulatory
limits \ref{sec:regulatory}. If we want to make high bandwidth links
over long distances the physics simply doesn't allow it within the
legal power limits.

So how does licensed spectrum work? This is discussed in
\cite{OfcomSpectrum}, and particularly the appendix called ``Fixed
link algorithm'' spells it out. This has to be worked out per link.

The first part of working it out is to answer the question, ``how much
spectrum do we want?'' The pricing is in 1 MHz increments and costs
£88/MHz per year. If we want to use 20MHz channels that costs
\pounds 1760. 56MHz channels, the widest commonly in use, is \pounds 
4928.

Next, what band do we want the license in? There is a discount applied
to some bands. The reason is because as frequency increases the
distances that are possible to achieve decrease. Again this is worked
out in section \ref{sec:link_analysis}. For 6 GHz, the discount is
26\%. For 11 and 13 GHz it is 57\%. For long links we have to use 6
GHz because much more than that and we need unreasonable output power
and unreasonably large antennas. So applying this, our 20 MHz channel
costs \pounds 1272.80 and our 56 MHz channel costs \pounds 3646.72.

There is a penalty charge for using lower frequencies where a link
would be possible with a higher one. That doesn't apply here.

There is also a charge relating to output power, which is called the
``availability factor''. Basically if we want to have a sufficiently
large margin in our link budget to overcome rare events link monsoons
we can have that, at a cost. We don't need that.

\begin{figure}[h]
  \begin{center}
    \input{licensing_costs}
  \end{center}
  \caption{Ofcom licensing costs, per link, in 6 and 11 GHz}
  \label{fig:license_costs}
\end{figure}

To begin with, we recommend 30MHz channels as that is sufficient to
support throughput of 150-200Mbps at 32QAM or 64QAM depending on
packet sizes.
